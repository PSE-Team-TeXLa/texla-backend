\section{Zielbestimmungen}
\label{sec:zielbestimmungen}

\texla{} soll es ermöglichen, \LaTeX"=Dokumente übersichtlich und hierarchisch darzustellen.
In dieser Form soll eine intuitive Benutzerschnittstelle zur Restrukturierung und Bearbeitung geboten werden.
Die Integration in bestehende Arbeitsabläufe soll über Git-Repositories und Overleaf erfolgen.

\subsection{Muss-Kriterien}
\label{subsec:muss-kriterien}

\criterium{Starten der Anwendung von der Konsole}{crt:cli}
% FA: Features der CLI: Starten, Ordner angeben, Kontrolliertes beenden

\criterium{Hierarchische grafische Darstellung von \LaTeX"=Dokumenten}{crt:hierarchy}
% FA: auch file includes unterstützen

\criterium{Erkennen von einfachen Strukturelementen}{crt:easy-elements}
% in Funktionen konkret, welche Elemente; Floats an der Stelle, an der sie im Code stehen; alle anderen als Codeblock;
% bestimmte Elemente als "bekannter Codeblock", d.h. es gibt ein Mapping von Environment auf benutzerfreundlicher Name,
% der dann als Kompaktform dienen kann (und ggf. zusätzlich angezeigt wird);
% unbekannte Inline-Makros müssen irgendwie ignoriert werden

\criterium{Erkennen von einfachen Formatierungsbefehlen}{crt:easy-format}

\criterium{Übersichtlichkeit durch automatisches Ein- und Ausblenden sowie durch Kompaktformen}{crt:clarity}
% ab durch kann man das auch in FAs auslagern

\criterium{Verschieben der Strukturelemente per Drag and Drop}{crt:dnd}
% FA: auch zwischen Ebenen

\criterium{Bearbeitung des \LaTeX"=Quelltextes von Strukturelementen mit dem Minieditor}{crt:minieditor}

\criterium{Hinzufügen von beliebigen Strukturelementen an beliebiger Stelle}{crt:add}
% Zwischenknöpfe + Minieditor, Minieditor eines anderen Elements, bekannte Strukturelemente werden erkannt und
% entsprechend angezeigt

\criterium{Persistentes Speichern von Zusatzinformationen als Kommentare in den \LaTeX"=Dateien\vspace*{-1em}}
{crt:meta-comments}
% in speziellen Kommentaren

\criterium{Benutzerdefinierter Export von \LaTeX"=Dateien}{crt:export}
% Settings panel / Profile

\clearpage

\criterium{Integration mit Git}{crt:git}
% regelmäßiges Synchronisieren, regelmäßiges Commiten nach Änderungen

\criterium{Direktlink zum entsprechenden Projekt in Overleaf}{crt:overleaf-link}
% auf konkrete Dateien geht nicht

\subsection{Kann-Kriterien}
\label{subsec:kann-kriterien}

\criteriumOptional{Konfigurierbare grafische Oberfläche}{crt:configurable-gui}
% z.B. Kommentare ausblenden

\criteriumOptional{Tastaturkürzel für häufige Operationen}{crt:keyboard}

\criteriumOptional{Hinzufügen von Notizen zu Strukturelementen}{crt:notes}

\criteriumOptional{Hinterlegen von benutzerdefinierten Kompaktformen an Strukturelementen}{crt:manual-compact-form}

\criteriumOptional{Syntax-Hervorhebung im Minieditor}{crt:syntax-highlighting}
% Minieditor?

\criteriumOptional{Code-Vervollständigung im Minieditor}{crt:code-completion}

\criteriumOptional{Möglichkeit zur einfachen Übersetzung der Anwendung}{crt:multilingual}
% FA: internationalization tool verwenden, Englisch anbieten

\criteriumOptional{Kompatibilität mit Browser-Extensions zur Autokorrektur}{crt:autocorrect}

\criteriumOptional{Automatische Zusammenfassung von Abschnitten durch KI}{crt:ai-tldrs}
% einen Knopf zur (de)aktivierung, ansonsten unsichtbar

\criteriumOptional{Unterstützung beim Schreiben durch KI}{crt:ai-rewriting}
% Vorschläge, Umschreiben nach Kommentar

\criteriumOptional{Bereitstellung als reinen Online-Dienst durch eine vorgeschaltete Login-Page}{crt:online}

\clearpage

\criteriumOptional{Feste Voreinstellungsprofile für den benutzerdefinierten Export}{crt:export-profiles}

\criteriumOptional{Export von PDF-Dateien}{crt:pdf-export}
% Voraussetzung: lokale Latex-Installation

\criteriumOptional{Verschieben von Listenelementen per Drag and Drop}{crt:lists}

\subsection{Abgrenzungskriterien}
\label{subsec:abgrenzungskriterien}

\criteriumNot{Die App ist kein vollständiger \LaTeX"=Parser oder -Compiler.}{crt:no-compiler}
Insbesondere werden keine benutzerdefinierten Makros unterstützt.

\criteriumNot{Die App muss mit fehlerhaften Dokumenten nicht umgehen können.}{crt:no-faulty-documents}
Insbesondere ist es möglich, dass der Benutzer im Minieditor invaliden oder nicht strikt hierarchischen
\LaTeX"=Quelltext (siehe~\ref{sec:produktleistungen}) schreibt und die Anwendung das resultierende Dokument nicht mehr
lesen kann.

\criteriumNot{Die App stellt keine eigene PDF-Vorschau zur Verfügung.}{crt:no-pdf-preview}

\criteriumNot{Die App stellt keine User-Verwaltung zur Verfügung, auch nicht für Git.}{crt:no-users}
