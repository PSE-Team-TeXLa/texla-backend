\section{Produktfunktionen}
\label{sec:produktfunktionen}

\subsection{Verpflichtende funktionale Anforderungen}
\label{subsec:verpflichtende-funktionale-anforderungen}

\texla{} soll folgende Funktionen unterstützen:

\functionality{Starten der Anwendung}{fnc:start}
\fulfills{crt:cli}

\texla{} lässt sich von der Konsole aus starten.
Dabei wird als Argument eine \LaTeX"=Datei angegeben.
\texla{} arbeitet anschließend auf dieser und den durch sie eingebundenen Dateien.

\functionality{Beenden der Anwendung}{fnc:exit}
\fulfills{crt:cli}

Über die GUI von \texla{} lässt sich das Programm kontrolliert beenden.
Dabei werden alle Änderungen lokal gespeichert und in das lokale Git-Repository sowie das Remote-Repository übernommen.

\functionality{Hierarchische Darstellung}{fnc:hierarchy}
\fulfills{crt:hierarchy}

Abschnitte in einem \LaTeX"=Dokument werden mit den Befehlen \verb|\part|, \verb|\chapter|, \verb|\section|,
\verb|\subsection|, \verb|\subsubsection|, \verb|\paragraph| und \verb|\subparagraph| auf verschiedenen Ebenen
definiert.
Basierend darauf stellt \texla{} das ausgewählte \LaTeX"=Dokument in hierarchischer Form dar.
Somit sind diese Abschnitte sogenannte Expandables.

In einer zweispaltigen Ansicht -- im Folgenden \enquote{Standardmodus} genannt -- werden dafür alle Elemente bis zu
einer gewünschten Ebene dargestellt.
Durch Auswahl eines Abschnittes bzw. durch eine entsprechende Schaltfläche in der GUI ist ein Wechsel zu einer tieferen
bzw. zu einer höheren Ebene möglich.

\functionality{Unterstützung inkludierter Dateien}{fnc:input}
\fulfills{crt:hierarchy}

In einer \LaTeX"=Datei können mittels \verb|input| weitere \LaTeX"=Dateien integriert werden.
Deren Inhalt wird ebenso in der GUI angezeigt.
Dabei entsteht für jede Verwendung von \verb|input| ein Strukturelement, dem auf tieferen Ebenen die jeweiligen
Strukturelemente der inkludierten Datei untergeordnet sind.
Somit ist \verb|input| ebenfalls ein Expandable.

\clearpage

\functionality{Strukturelemente in Blöcke aufteilen}{fnc:blocks}
\fulfills{crt:easy-elements}

Mittels \verb|\begin{...}| und \verb|\end{...}| können Umgebungen in \LaTeX"=Dokumenten genutzt werden.
\texla{} erkennt solche Umgebungen und stellt diese wie auch die genannten Abschnitte jeweils als separaten Block dar.
Umgebungen sind keine Expandables.
Ineinander verschachtelte Umgebungen werden also zu einem Block zusammengefasst und im Gegensatz zu Abschnitten nicht
auf mehrere Ebenen aufgeteilt.
Eine Ausnahme hiervon bildet die Umgebung \verb|document|.

\functionality{Bestimmte Strukturelemente rendern}{fnc:render}
\fulfills{crt:easy-elements}

Folgende Strukturelemente werden in gerenderter Form, vergleichbar mit der Ansicht im PDF-Dokument, angezeigt:

\begin{itemize}
  \item Bilder, definiert durch \verb|\includegraphics|
  \item Formeln, definiert durch \verb|\[ ... \]|, \verb|$$ ... $$| oder durch eine der Umgebungen \verb|displaymath|
  und \verb|equation|
\end{itemize}

Für alle anderen Strukturelemente reicht eine Darstellung in Form des \LaTeX"=Quelltextes innerhalb des jeweiligen
Blocks aus.

Sofern sich Elemente wie Bilder in Floats befinden, richtet sich die Position der Darstellung in \texla{} nach der
Position im Quelltext, nicht nach der Position im PDF-Dokument.

\functionality{Formatierung in Textblöcken}{fnc:format}
\fulfills{crt:easy-format}

Bei Textblöcken sowie bei Überschriften von Abschnitten werden die \LaTeX"=Befehle \verb|\textbf| und \verb|\textit|
berücksichtigt.
Entsprechende Teile des Textes werden fett bzw. kursiv hervorgehoben.

\functionality{Ein- und Ausblenden von Strukturelementen}{fnc:hide}
\fulfills{crt:clarity}

Wie in \functionalitylink{fnc:hierarchy} erläutert, stellt \texla{} nur Strukturelemente bis zur gewählten Ebene dar.
Alle Strukturelemente auf tieferen Ebenen werden ausgeblendet.
Beim Wechsel zwischen Ebenen werden automatisch die entsprechenden Elemente ein- bzw. ausgeblendet.

\clearpage

\functionality{Kompaktform von Strukturelementen}{fnc:compact}
\fulfills{crt:clarity}

Auf der linken Seite der zweispaltigen Ansicht von \texla{} werden die gleichen Expandables wie auf der rechten Seite
angezeigt.
Links werden die Elemente in kompakter Form dargestellt, rechts vollständig.

Bei Textblöcken entspricht die kompakte Darstellung den ersten Worten des Absatzes, bei Abschnitten der dazugehörigen
Überschrift, bei Bildern einer verkleinerten Anzeige des Bildes und bei sonstigen Strukturelementen dem Namen der
entsprechenden Umgebung.

Eine Ausnahme hiervon bilden Algorithmen, definiert durch eine der Umgebungen \verb|lstlisting|, \verb|minted|,
\verb|codebox| und \verb|algorithm|.
Diese werden unabhängig von der konkreten Umgebung als Algorithmus erkannt.
Die Kompaktform dieser Umgebungen entspricht dem Wort \enquote{Algorithmus} bzw. \enquote{algorithm}.

\functionality{Bereitstellung einer Graphansicht}{fnc:graph}
\fulfills{crt:hierarchy}

\texla{} bietet neben dem bisher beschriebenen Standardmodus einen \enquote{Graphmodus}, in dem alle Strukturelemente
auf allen Ebenen in Kompaktform angezeigt werden.

\functionality{Verschieben von Elementen per Drag and Drop}{fnc:drag-and-drop}
\fulfills{crt:dnd}

Im Standardmodus kann jedes Element in der rechten Spalte per Drag and Drop verschoben werden.
Der Inhalt des Strukturelements bleibt dabei unverändert.
Eine Verschiebung von der rechten in die linke Spalte ist nicht möglich.
Die von \LaTeX{} vorgegebene Grenze von maximal 7 Ebenen für Abschnitte kann beim Verschieben nicht überschritten
werden.

Im Graphmodus ist ebenso ein Verschieben beliebiger Strukturelemente an beliebige Stellen unter den bestehenden
Einschränkungen möglich.

\functionality{Bereitstellung eines Minieditors}{fnc:minieditor}
\fulfills{crt:minieditor}

Für jedes Strukturelement lässt sich im Standardmodus ein Minieditor öffnen.
Darin ist eine Bearbeitung des entsprechenden \LaTeX"=Quelltextes möglich.
Dieser Modus wird als \enquote{Bearbeitungsmodus} bezeichnet.

\clearpage

\functionality{Hinzufügen von Strukturelementen}{fnc:add}
\fulfills{crt:add}

Zwischen aufeinanderfolgenden Strukturelementen lässt sich im Standardmodus ein weiteres Strukturelement hinzufügen.
In der GUI existiert dafür eine entsprechende Schaltfläche an den jeweiligen Stellen.
Durch einen Klick darauf öffnet sich ein neuer Minieditor mit leerem Inhalt.
Der anschließend darin eingegebene \LaTeX"=Quelltext bildet das neue Strukturelement.

Alternativ lässt sich ein neues Strukturelement auch durch den jeweiligen \LaTeX"=Befehl innerhalb des Minieditors
eines bereits vorhandenen Elements erstellen.
Der ursprüngliche Block wird anschließend entsprechend in mehrere Blöcke aufgeteilt, ggf. auf verschiedenen Ebenen.

\functionality{Speichern von Zusatzinformationen in Kommentaren}{fnc:meta-comments}
\fulfills{crt:meta-comments}

Notwendige Zusatzinformationen werden innerhalb der \LaTeX"=Datei in gesonderten Kommentaren persistent gespeichert.
Diese Kommentare unterscheiden sich durch ein eindeutiges Präfix von regulären Kommentaren im \LaTeX"=Quelltext.

\functionality{Benutzerdefinierter Export}{fnc:export}
\fulfills{crt:export}

Das ausgewählte \LaTeX"=Dokument lässt sich exportieren.
Durch Einstellungsmöglichkeiten lassen sich einzelne Teile des Dokuments wie Kommentare im \LaTeX"=Quelltext vom Export
ausschließen.
Beim Export wird die ursprüngliche Ordner- und Dateistruktur beibehalten.

\functionality{Integration mit Git}{fnc:git}
\fulfills{crt:git}

\texla{} übernimmt Änderungen am \LaTeX"=Dokument stets in die entsprechenden lokalen Dateien.
Nach 5~\si{\second} ohne Änderungen erfolgt automatisch ein Commit in das lokale Git-Repository sowie ein Push in das
entsprechende Remote-Repository.

\functionality{Link zum Projekt in Overleaf}{fnc:overleaf-link}
\fulfills{crt:overleaf-link}

Die GUI bietet einen Direktlink zum entsprechenden Projekt in Overleaf, das mit dem Git-Repository verbunden ist.
Ein Direktlink zur aktuell geöffneten Datei wird nicht angeboten.

\clearpage

\subsection{Optionale funktionale Anforderungen}
\label{subsec:optionale-funktionale-anforderungen}

Zudem ergeben sich weitere optionale Funktionen für \texla:

\functionality{Konfigurierbare GUI}{fnc:configurable-gui}
\fulfills{crt:configurable-gui}

Die GUI von \texla{} lässt sich konfigurieren.
Beispielsweise lassen sich Kommentare im \LaTeX"=Quelltext ausblenden.

\functionality{Tastaturkürzel}{fnc:keyboard}
\fulfills{crt:keyboard}

\texla{} bietet Tastaturkürzel für einzelne Operationen, \zB{} das Wechseln zwischen Standardmodus und Graphmodus, an.

\functionality{Hinzufügen von Notizen}{fnc:notes}
\fulfills{crt:notes}

Zu jedem Strukturelement lässt sich über eine entsprechende Schaltfläche in der GUI eine Notiz hinzufügen.

\functionality{Benutzerdefinierte Kompaktformen von Strukturelementen}{fnc:manual-compact-form}
\fulfills{crt:manual-compact-form}

Abweichend von \functionalitylink{fnc:compact} lassen sich benutzerdefinierte Kompaktformen für Strukturelemente
festlegen.
Dafür existiert neben jedem Strukturelement eine entsprechende Schaltfläche in der GUI.

\functionality{Syntax-Hervorhebung im Minieditor}{fnc:syntax-highlighting}
\fulfills{crt:syntax-highlighting}

Im Minieditor eines Strukturelements werden \LaTeX"=Befehle farblich hervorgehoben.

\functionality{Code-Vervollständigung im Minieditor}{fnc:code-completion}
\fulfills{crt:code-completion}

Im Minieditor eines Strukturelements wird eine Code-Vervollständigung angeboten.
Beim Tippen eines \LaTeX"=Befehls werden entsprechend passende Befehle angezeigt.
Durch Auswählen eines Vorschlags wird der Befehl an der aktuellen Stelle eingefügt.

\clearpage

\functionality{Übersetzung der Anwendung}{fnc:multilingual}
\fulfills{crt:multilingual}

\texla{} bietet die Möglichkeit, weitere Sprachen für die Anwendung einzubinden.
Die Standardsprache ist Englisch.

\functionality{Kompatibilität mit Browser-Extensions zur Autokorrektur}{fnc:autocorrect}
\fulfills{crt:autocorrect}

\texla{} stellt die Kompatibilität mit einzelnen Browser-Extensions sicher, die eine Autokorrektur bezüglich
Rechtschreibung und Grammatik bieten.

\functionality{Automatische Zusammenfassung von Abschnitten durch KI}{fnc:ai-tldrs}
\fulfills{crt:ai-tldrs}

\texla{} bietet die Möglichkeit, KI-generierte Zusammenfassungen für Abschnitte als Kompaktformen zu verwenden.
Diese Funktion lässt sich durch eine entsprechende Schaltfläche in der GUI aktivieren.

\functionality{Unterstützung beim Schreiben durch KI}{fnc:ai-rewriting}
\fulfills{crt:ai-rewriting}

\texla{} bietet die Möglichkeit, Abschnitte durch KI-Unterstützung umzuschreiben.
Dazu kann ein Kommentar angegeben werden, der als Anfrage an die KI weitergegeben wird.

\functionality{Bereitstellung als reinen Online-Dienst}{fnc:online}
\fulfills{crt:online}

\texla{} wird als reiner Online-Dienst bereitgestellt, anstatt als lokale Installation verwendet zu werden.

\functionality{Voreinstellungsprofile für den benutzerdefinierten Export}{fnc:export-profiles}
\fulfills{crt:export-profiles}

Wie in \functionalitylink{fnc:export} beschrieben, bietet \texla{} beim Export noch Einstellungsmöglichkeiten.
Dafür lassen sich einzelne Profile auswählen.

\functionality{Export von PDF-Dateien}{fnc:pdf-export}
\fulfills{crt:pdf-export}

\texla{} bietet die Möglichkeit, das aktuelle Dokument als PDF-Datei zu exportieren.

\clearpage

\functionality{Verschieben von Listenelementen per Drag and Drop}{fnc:lists}
\fulfills{crt:lists}

Elemente innerhalb einer Listenumgebung wie \verb|itemize| oder \verb|enumerate| lassen sich per Drag and Drop in der
Reihenfolge umordnen.
