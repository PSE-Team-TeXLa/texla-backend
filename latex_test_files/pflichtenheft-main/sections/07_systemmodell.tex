\section{Systemmodell}
\label{sec:systemmodell}

\paragraph{Lokalität und Git}
Zur Benutzung muss die Anwendung lokal installiert werden.
Sie wird dann auf einer Datei in einem Ordner des lokalen Dateisystems ausgeführt,
der gleichzeitig ein Git-Repository sein kann.
Das heißt, solange die Anwendung nicht läuft, sind alle persistenten Daten in den lokalen Dateien auf dem Endgerät
gespeichert.
Sie werden regelmäßig per Commit in das lokale Git-Repository und per Push auf ein ggf. vorhandenes Remote-Repository
übertragen.

\paragraph{Overleaf}
Gegebenenfalls werden Links zu Overleaf angeboten --
diese dienen jedoch nur der Weiterleitung des Users und nicht zur Kommunikation zwischen den beiden Diensten.

\paragraph{Backend}
Für das Backend inklusive \textit{CLI} wird die Programmiersprache Rust verwendet.
Für den automatischen Import werden ein \textit{Directory Watcher} und Git verwendet.
Für das Parsen von und Arbeiten auf \LaTeX"=Dokumenten wird ein \textit{abstrakter Syntaxbaum} (\textit{AST})
verwendet.
Dieser AST stellt zu jeder Zeit den aktuell gültigen Stand der Dinge dar.

\paragraph{Frontend}
Das Frontend ist eine Web-Applikation, die im Browser ausgeführt und angezeigt wird.
Für das Frontend werden TypeScript und das Web-Framework \textit{Svelte} verwendet.
Das Rendering des \LaTeX"=Quelltextes im AST geschieht im Frontend.

\paragraph{Kommunikation}
Für die Kommunikation zwischen Backend und Frontend werden eine \textit{REST-API} und das JSON-Dateiformat
verwendet.
Insbesondere verschickt das Frontend an das Backend Befehle zur Manipulation des AST und
das Backend bietet eine Schnittstelle zum Lesen des aktuellen AST.
