\section{Produktleistungen}
\label{sec:produktleistungen}

\texla{} garantiert unter den folgenden Voraussetzungen, die aufgeführten Anforderungen zu erfüllen:

\begin{itemize}
  \item Die zu bearbeitenden \verb|.tex|"=Dateien befinden sich lokal auf dem Zielrechner in einem Git-Repository,
  das mit einem Remote-Repository verknüpft ist.
  \item Die Dateigrößen aller zu bearbeitenden \verb|.tex|"=Dateien überschreiten in der Summe nicht
  100~\si{\kilo\byte}.
  \item Die zu bearbeitenden \verb|.tex|"=Dateien sind \textbf{strikt hierarchisch} aufgebaut.
  \item Die zu bearbeitenden \verb|.tex|"=Dateien verfügen über höchstens 5 Ebenen der hierarchischen Verschachtelung.
  \item Abschnitte werden in den zu bearbeitenden \verb|.tex|"=Dateien nicht innerhalb von \LaTeX"=Umgebungen
  geöffnet.
  Die einzige Ausnahme ist die Umgebung \verb|document|.
\end{itemize}

\textbf{Strikt hierarchisch} beschreibt ein \LaTeX"=Dokument, für das Folgendes gilt:

\begin{itemize}
  \item Jeder Abschnitt, der geöffnet wird, wird innerhalb des ihm übergeordneten Abschnitts geöffnet.
  Die höchste Abschnittsebene des Dokuments ist hiervon ausgenommen.
  Folgende Abschnittsebenen werden unterstützt:
  \verb|part|, \verb|chapter|, \verb|section|, \verb|subsection|, \verb|subsubsection|, \verb|paragraph| und
  \verb|subparagraph|.

  \textbf{Beispiel:}
  \begin{verbatim}
\section{Heading1}
\subsection{Subheading1}
  \end{verbatim}

  \textbf{Gegenbeispiel:}
  \begin{verbatim}
\section{Heading1}
\subsubsection{Subsubheading1}
  \end{verbatim}

  \clearpage

  \item \verb|\input| kann auf jeder Ebene geöffnet werden.
  Für jeden Abschnitt und \verb|\input| ist entweder der komplette Abschnitt innerhalb des \verb|\input| oder der
  komplette \verb|\input| innerhalb des Abschnitts.

  \textbf{Beispiel:}
  \begin{verbatim}
\section{Heading1}
\input{file1}
\section{Heading3}

file1.tex:
\section{Heading2}
  \end{verbatim}

  \textbf{Gegenbeispiel 1:}
  \begin{verbatim}
\section{Heading1}
\input{file1}
Some text

file1.tex:
\section{Heading2}
  \end{verbatim}

  \textbf{Gegenbeispiel 2:}
  \begin{verbatim}
\section{Heading1}
\input{file1}
\section{Heading3}

file1.tex:
Some text
\section{Heading2}
  \end{verbatim}
\end{itemize}
