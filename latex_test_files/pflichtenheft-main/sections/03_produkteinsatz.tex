\section{Produkteinsatz}
\label{sec:produkteinsatz}

Die Anwendung dient zur Restrukturierung und Bearbeitung von \LaTeX"=Dokumenten.
Sie hilft dabei, den Überblick über große Dokumente zu behalten und strukturelle Änderungen vorzunehmen,
ohne \verb|.tex|"=Dateien bearbeiten zu müssen.
Die Anwendung ermöglicht die Synchronisation mit existierenden Overleaf-Projekten.
% Import mit einfachem clone zwar auch sehr gut denkbar, aber steht erstmal in keinem Kriterium/FA

\subsection{Zielgruppe}
\label{subsec:zielgruppe}

Die Zielgruppe der Anwendung sind \LaTeX"=Benutzer.
In erster Linie sind das Wissenschaftler und andere Menschen, welche im akademischen Kontext arbeiten.
Auch Organisationen, welche technische Dokumente in \LaTeX{} erstellen, können von der Anwendung profitieren.
Die Anwendung richtet sich nicht explizit an \LaTeX"=Anfänger, da sie die Benutzung von \LaTeX{} nicht per se
vereinfacht, sondern bestimmte Abläufe effizienter und angenehmer macht.

\subsection{Betriebsbedingungen}
\label{subsec:betriebsbedingungen}

Die Anwendung folgt einem Client-Server-Modell.
Der Benutzer installiert die Server-Komponente auf seinem Computer und interagiert mit der Anwendung mittels einem
Frontend in seinem Webbrowser.
Eine Internetverbindung ist für einige Funktionen notwendig.

Eine Portierung der Server-Komponente auf einen zentral gehosteten Webserver ist möglich, jedoch nicht notwendig.
