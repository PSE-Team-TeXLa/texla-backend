\clearpage

\section{Testfallszenarien}
\label{sec:testfallszenarien}
\subsection{Funktionale Testfälle}
\label{subsec:funktionale-testfaelle}
Die folgenden Testfälle testen die verpflichtenden funktionalen Anforderungen:

\test{Starten der Anwendung}{tst:start-01}

\begin{itemize}
\item Vorbedingung: Dateistruktur erfüllt die Voraussetzungen (siehe~\ref{sec:produktleistungen}).
  \item Aktion: Programmstart durch Übergabe eines Dateipfades als Argument an die ausführbare Datei.
  \item Nachbedingung: Standardansicht in der höchsten Ebene der Anwendung.

\end{itemize}
\tests{fnc:start}

\test{Beenden der Anwendung}{tst:stop-01}
\tests{fnc:exit}

\begin{itemize}
\item Vorbedingung: Die Anwendung ist aktiv und befindet sich in irgendeinem Ansichtsmodus.
  \item Aktion: Klicken des Exit-Symbols in der unteren linken Ecke der Sidebar.
  \item Nachbedingung: Die Anwendung wird geschlossen.
  Die lokalen und Remote-Repositories sind synchron mit den vom Benutzer in \texla{} vorgenommenen Änderungen.

\end{itemize}
\test{Hierarchische Darstellung}{tst:hierarchy-01}
\tests{fnc:hierarchy}

\begin{itemize}
\item Vorbedingung: Das \texla"=Pflichtenheft ist lokal verfügbar.
  \item Aktion: Öffnen der \texla"=Anwendung mit dem Dateipfad zum \texla"=Pflichtenheft.
  \item Nachbedingung: Alle unterstützten Strukturelemente wurden erkannt und korrekt dargestellt.

\end{itemize}
\clearpage

\test{Unterstützung inkludierter Dateien}{tst:input-01}
\tests{fnc:input}

\begin{itemize}
\item Vorbedingung: \LaTeX"=Datei mit \verb|/input| steht lokal zur Verfügung.
  \texla{} ist gestartet.
  \item Aktion: Navigation zum Strukturelement des \verb|/input|-Makros.
  \item Nachbedingung: Die Datei wurde erkannt und wird angezeigt.

\end{itemize}
\test{Strukturelemente in Blöcke aufteilen}{tst:blocks-01}
\tests{fnc:blocks}

\begin{itemize}
\item Vorbedingung: \LaTeX"=Datei enthält \verb|/begin{...} ... /end{...}|.
  \texla{} ist gestartet.
  \item Aktion: Navigation zur definierten Umgebung.
  \item Nachbedingung: Das Umgebungs-Strukturelement wurde erkannt und korrekt angezeigt.

\end{itemize}
\test{Rendering von Bildern}{tst:render-01}
\tests{fnc:render}

\begin{itemize}
\item Vorbedingung: \LaTeX"=Datei mit \verb|/includegraphics| steht lokal zur Verfügung.
  \texla{} ist gestartet.
  \item Aktion: Navigation zum definierten Bild.
  \item Nachbedingung: Das Bild-Strukturelement wurde erkannt und korrekt gerendert.

\end{itemize}
\test{Rendering von mathematischen Formeln}{tst:render-02}
\tests{fnc:render}

\begin{itemize}
\item Vorbedingung: \LaTeX"=Datei mit \verb|/begin{displaymath} ... /end{displaymath}| steht lokal zur Verfügung.
  \texla{} ist gestartet.
  \item Aktion: Navigation zur definierten Formel.
  \item Nachbedingung: Die mathematische Formel wurde erkannt und korrekt gerendert.

\end{itemize}
\clearpage

\test{Formatierung in Textblöcken}{tst:format-01}
\tests{fnc:format}

\begin{itemize}
\item Vorbedingung: \LaTeX"=Datei mit \verb|/textbf| und \verb|/textit| steht lokal zur Verfügung.
  \texla{} ist gestartet.
  \item Aktion: Navigation zum enthaltenden Strukturelement.
  \item Nachbedingung: Fetter und kursiver Text wurde gerendert.

\end{itemize}
\test{Ein- und Ausblenden sowie Komprimieren von Strukturelementen}{tst:clarity-01}
\tests{fnc:hide}
\tests{fnc:compact}

\begin{itemize}
\item Vorbedingung: \LaTeX"=Datei mit mehrstufiger Gliederung (\zB{} \verb|/section| und \verb|/subsection|) steht
  lokal zur Verfügung.
  \texla{} ist gestartet.
  \item Aktion: Navigation zum Strukturelement.
  Ein Strukturelement wird in der Strukturspalte angeklickt.
  \item Nachbedingung: Die Überschriften der Strukturelemente werden kompakt in der Strukturspalte angezeigt.
  Verborgene und erweiterbare Elemente werden kompakt in der Lesespalte angezeigt.

\end{itemize}
\test{Erkennen von Algorithmen}{tst:compact-02}
\tests{fnc:compact}

\begin{itemize}
\item Vorbedingung: \LaTeX"=Datei mit Algorithmus-Umgebung steht lokal zur Verfügung.
  \texla{} ist gestartet.
  \item Aktion: Navigation zur Ebene oberhalb des definierten Algorithmus.
  \item Nachbedingung: Das Algorithmus-Strukturelement wird in Kompaktform durch den Text \enquote{Algorithmus} bzw.
  \enquote{algorithm} repräsentiert.

\end{itemize}
\test{Bereitstellung einer Graphansicht}{tst:graph-01}
\tests{fnc:graph}

\begin{itemize}
\item Vorbedingung: \texla{} ist gestartet auf validem \LaTeX"=Dokument.
  \item Aktion: Klicken auf den Ansichtwechsel-Button.
  \item Nachbedingung: Die Graphansicht wird angezeigt.

\end{itemize}
\test{Legale Verschiebung von Elementen per Drag and Drop}{tst:dragdrop-01}
\tests{fnc:drag-and-drop}

\begin{itemize}
\item Vorbedingung: \texla{} ist gestartet auf validem \LaTeX"=Dokument.
  \item Aktion: Navigation zum Strukturelement.
  In der Lesespalte wird ein Element angeklickt und nach oben oder unten, vor oder nach ein anderes Element
  verschoben.
  \item Nachbedingung: Das Element steht an der gewünschten Position.

\end{itemize}
\test{Illegale Verschiebung von Elementen per Drag and Drop}{tst:dragdrop-02}
\tests{fnc:drag-and-drop}

\begin{itemize}
\item Vorbedingung: \LaTeX"=Datei mit mehrstufiger Gliederung (\zB{} \verb|/section| und \verb|/subsection|) steht
  lokal zur Verfügung.
  \texla{} ist gestartet.
  \item Aktion: Navigation zum Strukturelement.
  In der Lesespalte wird ein Element angeklickt und zur Strukturspalte verschoben.
  \item Nachbedingung: Das Element behält seine ursprüngliche Position.

\end{itemize}
\test{Bereitstellung eines Minieditors}{tst:minieditor-01}
\tests{fnc:minieditor}

\begin{itemize}
\item Vorbedingung: \texla{} ist gestartet auf validem \LaTeX"=Dokument mit mindestens einem textuellen Absatz.
  \item Aktion: Navigation zum und Anklicken eines textuellen Absatzes im Arbeitsbereich.
  Anklicken des Absatzes in der Lesespalte.
  Navigation zum Kontextmenü und Anklicken des Bearbeitungsmodus.
  \item Nachbedingung: Die linke Spalte wird schmaler angezeigt.
  Der Minieditor für den ausgewählten Absatz wird auf der rechten Seite angezeigt.

\end{itemize}
\clearpage

\test{Hinzufügen von Abschnitten}{tst:add-01}
\tests{fnc:add}

\begin{itemize}
\item Vorbedingung: \texla{} ist gestartet auf validem \LaTeX"=Dokument.
  \item Aktion: Navigation zur Lesespalte.
  In der Standardansicht hinter oder zwischen beliebigen Elementen hovern.
  Klicken des H-Button, wodurch ein Popup erscheint.
  Hinzufügen einer Überschrift.
  \item Nachbedingung: Das neue Element wird in der Standardansicht angezeigt.

\end{itemize}
\test{Hinzufügen von \LaTeX"=Blöcken}{tst:add-02}
\tests{fnc:add}

\begin{itemize}
\item Vorbedingung: \texla{} ist gestartet auf validem \LaTeX"=Dokument.
  \item Aktion: Navigation zur Lesespalte.
  In der Standardansicht hinter oder zwischen beliebigen Elementen hovern.
  Klicken des T-Button, wodurch \texla{} in den Bearbeitungsmodus wechselt.
  Bearbeitung und Speichern.
  \item Nachbedingung: Der neue \LaTeX"=Block wird in der Standardansicht angezeigt.

\end{itemize}
\test{Verschmelzen von zwei Textabsätzen}{tst:add-03}
\tests{fnc:add}

\begin{itemize}
\item Vorbedingung: \LaTeX"=Datei mit mindestens zwei textuellen Absätzen auf gleicher Ebene steht lokal zur
  Verfügung.
  \texla{} ist gestartet.
  \item Aktion: Navigation zur Lesespalte.
  Bearbeiten im Minieditor des unteren von zwei textuellen Absätzen.
  Klicken der Rücktaste am Anfang des Absatzes.
  \item Nachbedingung: Die beiden Strukturelemente wurden verschmolzen.

\end{itemize}
\clearpage

\test{Benutzerdefinierter Export}{tst:export-01}
\tests{fnc:export}

\begin{itemize}
\item Vorbedingung: \texla{} ist gestartet auf validem \LaTeX"=Dokument.
  \item Aktion: Navigation zur Sidebar und Auswahl der Export-Option.
  Auswahl von Exportmöglichkeiten in einem Pop-up-Fenster der GUI.
  \item Nachbedingung: Die exportierte Datei steht mit den ausgewählten Optionen zur Verfügung.

\end{itemize}
\test{Integration mit Git}{tst:git-01}
\tests{fnc:git}

\begin{itemize}
\item Vorbedingung: \texla{} ist gestartet.
  \item Aktion: Verändern der \LaTeX"=Datei mit \texla{}.
  Abwarten von 5~\si{\second}.
  \item Nachbedingung: Ein Commit wurde im lokalen Repository durchgeführt.
  Ein Push wurde zum Remote-Repository durchgeführt.

\end{itemize}
\test{Link zum Projekt in Overleaf}{tst:overleaf-01}
\tests{fnc:overleaf-link}

\begin{itemize}
\item Vorbedingung: \texla{} ist gestartet auf validem \LaTeX"=Dokument.
  \item Aktion: Navigation zur Sidebar und Auswahl der Overleaf-Option.
  \item Nachbedingung: Der Browser öffnet sich mit dem Projekt in Overleaf.

\end{itemize}
Die folgenden Testfälle testen die optionalen funktionalen Anforderungen:

\test{Konfigurierbare graphische Oberfläche}{tst:gui-01}
\tests{fnc:configurable-gui}

\begin{itemize}
\item Vorbedingung: \texla{} ist gestartet auf validem \LaTeX"=Dokument mit \LaTeX"=Kommentaren.
  \item Aktion: Navigation zur Sidebar und Auswahl der Option \enquote{Ausblenden von \LaTeX"=Kommentaren}.
  \item Nachbedingung: \LaTeX"=Kommentare werden nicht mehr angezeigt.

\end{itemize}
\test{Tastaturkürzel für häufige Operationen}{tst:keyboard}
\tests{fnc:keyboard}

\begin{itemize}
\item Vorbedingung: \texla{} ist gestartet auf validem \LaTeX"=Dokument.
  \item Aktion: Drücken der Tastenkombination zum Wechseln zwischen Modi.
  \item Nachbedingung: \texla{} wechselt zwischen Standard- und Graphmodus.

\end{itemize}
\test{Hinzufügen von Notizen}{tst:notes-01}
\tests{fnc:notes}
\tests{fnc:meta-comments}

\begin{itemize}
\item Vorbedingung: \LaTeX"=Datei mit mindestens einem Strukturelement steht lokal zur Verfügung.
  \texla{} ist gestartet.
  \item Aktion: Navigation zum Strukturelement.
  Öffnen des Kontextmenüs zu einem Strukturelement.
  Auswahl der Notiz-Option und Bearbeitung in einem Pop-up-Fenster der GUI.
  \item Nachbedingung: Die Notiz wird als Meta-Kommentar im Quelltext gespeichert.

\end{itemize}
\test{Benutzerdefinierte Kompaktformen von Strukturelementen}{tst:compact-01}
\tests{fnc:manual-compact-form}

\begin{itemize}
\item Vorbedingung: \LaTeX"=Datei mit mindestens einem Strukturelement steht lokal zur Verfügung.
  \texla{} ist gestartet.
  \item Aktion: Navigation zum Strukturelement.
  Öffnen des Kontextmenüs zu einem Strukturelement.
  Auswahl der Kompaktform-Option und Eingeben einer Zusammenfassung.
  \item Nachbedingung: Das Strukturelement wird an den entsprechenden Stellen in der neuen Kompaktform angezeigt.

\end{itemize}
\clearpage

\test{Syntax-Hervorhebung im Texteditor}{tst:syntax-highlighting-01}
\tests{fnc:syntax-highlighting}

\begin{itemize}
\item Vorbedingung: \LaTeX"=Datei mit mindestens einem Strukturelement steht lokal zur Verfügung.
  \texla{} ist gestartet.
  \item Aktion: Navigation zum Strukturelement.
  Öffnen des Kontextmenüs zu diesem Strukturelement.
  Auswahl der Bearbeiten-Option.
  \item Nachbedingung: Die \LaTeX"=Befehle des Strukturelements werden im Minieditor farblich hervorgehoben.

\end{itemize}
\test{Code-Vervollständigung im Texteditor}{tst:gcode-completion}
\tests{fnc:code-completion}

\begin{itemize}
\item Vorbedingung: \LaTeX"=Datei mit mindestens einem textuellen Strukturelement steht lokal zur Verfügung.
  \texla{} ist gestartet.
  \item Aktion: Navigation zum Strukturelement.
  Öffnen des Kontextmenüs zu diesem Strukturelement.
  Wechsel in den Bearbeitungsmodus.
  Eingabe von \LaTeX"=Befehlen.
  \item Nachbedingung: \LaTeX"=Befehle werden automatisch vervollständigt.

\end{itemize}
\test{Möglichkeit zur einfachen Übersetzung der Anwendung}{tst:multilingual}
\tests{fnc:multilingual}

\begin{itemize}
\item Vorbedingung: \texla{} ist gestartet.
  \item Aktion: Navigation zur Sidebar, Anklicken der Option \enquote{Sprache} und Auswahl einer bestimmten Sprache.
  \item Nachbedingung: Die Sprache der Anwendung wurde zur ausgewählten Sprache geändert.

\end{itemize}
\clearpage

\test{Kompatibilität mit Browser-Extensions zur Autokorrektur}{tst:autocorrect}
\tests{fnc:autocorrect}

\begin{itemize}
\item Vorbedingung: \texla{} ist gestartet.
  Die Browser-Extension LanguageTool ist installiert und aktiv.
  \item Aktion: Im Bearbeitungsmodus wird Text eingegeben, der bezüglich Grammatik oder Rechtschreibung fehlerhaft
  ist.
  \item Nachbedingung: Fehlerhafte Teile des Textes werden erkannt und entsprechend hervorgehoben.

\end{itemize}
\test{Automatische Zusammenfassung von Abschnitten durch KI}{tst:ai-tldrs}
\tests{fnc:ai-tldrs}

\begin{itemize}
\item Vorbedingung: \texla{} ist gestartet auf validem \LaTeX"=Dokument.
  \item Aktion: Navigation zur Sidebar und Auswahl der Option \enquote{KI-Zusammenfassung}.
  \item Nachbedingung: Die einzelnen Zusammenfassungen pro Abschnitt wurden automatisch generiert.

\end{itemize}
\test{Unterstützung beim Schreiben durch KI}{tst:ai-rewriting}
\tests{fnc:ai-rewriting}

\begin{itemize}
\item Vorbedingung: \texla{} ist gestartet auf validem \LaTeX"=Dokument.
  \item Aktion: Navigation zur Sidebar und Auswahl eines Strukturelements.
  Wechsel in den Bearbeitungsmodus.
  Eingabe einer Anfrage an die KI.
  \item Nachbedingung: Die KI hat eine Antwort zur Anfrage geschrieben.

\end{itemize}
\test{Bereitstellung als reinen Online-Dienst}{tst:online}
\tests{fnc:online}

\begin{itemize}
\item Vorbedingung: Internetverbindung ist vorhanden.
  \item Aktion: \texla{} wird im Browser geöffnet.
  \item Nachbedingung: Dokumente können mit \texla{} online bearbeitet werden.

\end{itemize}
\clearpage

\test{Voreinstellungsprofile für den benutzerdefinierten Export}{tst:export-profiles}
\tests{fnc:export-profiles}

\begin{itemize}
\item Vorbedingung: \texla{} ist gestartet.
  \item Aktion: Navigation zur Sidebar.
  Anklicken der Option \enquote{Export} und Auswahl eines voreingestellten Profils.
  \item Nachbedingung: Die Export-Einstellungen wurden entsprechend des Profils geändert.

\end{itemize}
\test{Export von PDF-Dateien}{tst:pdf-export}
\tests{fnc:pdf-export}

\begin{itemize}
\item Vorbedingung: \texla{} ist gestartet auf validem \LaTeX"=Dokument.
  \item Aktion: Navigation zur Sidebar und Auswahl der Export-Option.
  \item Nachbedingung: Das \LaTeX"=Dokument wurde als PDF exportiert.

\end{itemize}
\test{Verschieben von Listenelementen per Drag and Drop}{tst:list-dnd}
\tests{fnc:lists}

\begin{itemize}
\item Vorbedingung: \texla{} ist gestartet auf validem \LaTeX"=Dokument mit einer \verb|itemize|-Umgebung mit
  mindestens zwei Elementen.
  \item Aktion: Navigation zur Liste.
  Verschieben eines Listeneintrags per Drag and Drop.
  \item Nachbedingung: Listeneinträge erscheinen in neuer Reihenfolge.

\end{itemize}
\subsection{Szenarien}
\label{subsec:tests-scenarios}
Anmerkung: Jegliche Ähnlichkeit zu real existierenden Personen ist rein zufällig.

\subsubsection{Erweiterung und Verwaltung eines bestehenden \LaTeX"=Projekts}
Der Benutzer Paul hat das Ziel, sein umfangreiches \LaTeX"=Projekt zu erweitern und zu verwalten.
Bedauerlicherweise hat er aufgrund der Vielzahl an Dateien den Überblick verloren und einige \LaTeX"=Befehle sind
ihm entfallen, was sein Verständnis der einzelnen Abschnitte erschwert.

Paul navigiert zum lokalen Dateipfad seines Projekts und startet die \texla"=Anwendung mit einem Konsolenbefehl.
Nun befindet er sich in der Hauptansicht der Anwendung in seinem Browser und erblickt auf der linken
Seite die Sidebar.

Zunächst möchte Paul einen Gesamtüberblick über das Projekt gewinnen.
Hierzu wechselt er über die Sidebar in den Graphmodus und betrachtet die Baumansicht aller Elemente.
Dabei bemerkt er, dass ein Abschnitt zu groß geworden ist, weshalb er einen Unterabschnitt zu einem separaten Abschnitt
umwandeln will.
Mithilfe der Drag-and-Drop-Funktion passt er die Anordnung des Unterabschnitts an und wechselt dann zurück in den
Standardmodus.

Paul sieht nun auf der linken Seite eine kompakte Strukturspalte
mit Überschriften, die den verschiedenen Abschnitten zugeordnet sind.
Nun möchte er einen bestimmten Abschnitt bearbeiten und klickt hierzu auf die entsprechende kompakte Überschrift.
In der Lesespalte auf der rechten Seite werden ihm nun alle Elemente angezeigt, die zu diesem Abschnitt
gehören, sowie alle projizierten Elemente aus vorhergehenden Ebenen.
Er wählt ein Textelement und klickt den Bearbeitungsbutton.
Im Bearbeitungsmodus führt er die gewünschten Änderungen durch und speichert diese.

Im Anschluss stellt Paul fest, dass die Reihenfolge der Elemente optimiert werden könnte.
Er nutzt erneut die Drag-and-Drop-Funktion, um eine Überschrift hinter eine andere zu verschieben.

Zufrieden mit seinen Anpassungen schließt Paul die Anwendung.
All seine Änderungen wurden sowohl lokal als auch remote gespeichert.

\subsubsection{Unterstützung bei der Strukturverbesserung eines \LaTeX"=Projekts}
Benutzer Max verfolgt das Ziel, seinem Kommilitonen Paul bei der Verbesserung der Struktur seines
\LaTeX"=Projekts zu unterstützen.
Dafür setzt er die \texla"=Anwendung ein, um Paul hilfreiche Notizen zu hinterlassen.

Max betrachtet zunächst die Überschriften in der Strukturspalte und durchläuft diese systematisch.
Dabei bemerkt er, dass einige Abschnitte stilistisch inkorrekt formatiert sind.
Da es Max ein Anliegen ist, dass Paul einen korrekten \LaTeX"=Stil erlernt, entscheidet
er sich dagegen, das Problem eigenhändig zu beheben.

Stattdessen wählt er die Überschrift des betroffenen Elements und navigiert zur Lesespalte.
Er bewegt den Cursor über einen Abschnitt und wählt die Option des Minieditors.
Hier hinterlässt er für Paul wichtige Bearbeitungshinweise in Form von \LaTeX"=Kommentaren.

Schließlich beendet Max die Anwendung und teilt Paul die hinterlassenen Bearbeitungshinweise mit.

\subsubsection{Verbesserung von Algorithmen-Vorlesungsfolien}
Benutzer Linus, der an einer Universität als Tutor tätig ist, strebt danach, die \LaTeX"=Folien für die
Algorithmen-Vorlesung zu verbessern.
Sein Lehrstuhl bevorzugt die Nutzung von Overleaf, auf welchem auch die aktuellen Folien abgelegt sind.
Linus erkennt Optimierungspotenzial bei einigen Pseudocode-Algorithmen.
Allerdings sind die vorhandenen Pseudocode-Algorithmen aufgrund des unorganisierten \LaTeX"=Repositorys schwer
auffindbar.

Daher entscheidet er sich für den Einsatz von \texla{}.
Linus wechselt in den Graphmodus und erhält einen Überblick
über verschiedene Algorithmen-Blöcke in Form eines Baumes.
Durch die Auswahl eines Algorithmus-Blocks wechselt er in den
Standardmodus, und der ausgewählte Algorithmus ist nun in der Lesespalte auf der rechten Seite zu finden.
Mithilfe des Bearbeitungsmodus und des Minieditors optimiert er diese Blöcke.

Während der Bearbeitung kommt ihm eine brillante Idee.
Linus erstellt ein neues Element zwischen zwei bestehenden Abschnitten und bringt seine Idee dort unter.
Er möchte noch ein Bild zur Illustration beifügen, das er in den Ordner der \LaTeX"=Datei legt und noch in demselben
Minieditor einbindet.
Nach der Bestätigung im Minieditor werden der neue Absatz und das Bild als zwei getrennte Strukturelemente angezeigt.

Nach Abschluss der Bearbeitung möchte er seine Idee mit einem externen Freund von der Universität teilen.
Er stellt jedoch fest, dass er nicht die Berechtigung hat, ihn zum Repository des Lehrstuhls einzuladen.
Daher entscheidet er sich, das \LaTeX"=Dokument zu exportieren.
In den Exporteinstellungen wählt Linus aus, dass alles exportiert werden soll.
Schließlich sendet er die Datei an seinen Freund.

