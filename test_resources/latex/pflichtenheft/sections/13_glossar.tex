\clearpage

\section{Glossar}
\label{sec:glossar}

% für Begriffe, die bei uns besondere Bedeutung haben
\setlength{\extrarowheight}{2em}

\begin{longtable}{>{\bfseries}rp{9cm}}
  Absatz &
  Ein durch Leerzeilen abgetrenntes Stück Text. \\

  Abschnitt &
  Gibt Elementen einen semantischen Rahmen.
  Entspricht in \LaTeX{} \verb|part|, \verb|chapter|, \verb|section|, \verb|subsection|, \verb|subsubsection|,
  \verb|paragraph| oder \verb|subparagraph|. \\

  Arbeitsbereich &
  Bereich rechts von der Sidebar, welcher zur Restrukturierung genutzt wird.
  Der Bereich kann zwischen dem Standardmodus, dem Graphmodus und dem Bearbeitungsmodus wechseln. \\

  AST &
  Abstract Syntax Tree, abstrakter Syntaxbaum.
  Eine Datenstruktur, die typischerweise zum Speichern von und Arbeiten auf strukturell verschachtelten Dokumenten
  verwendet wird. \\

  CLI &
  Command Line Interface, Befehlszeilenschnittstelle. \\

  Element, Strukturelement &
  Das Dokument wird in Elemente gegliedert, wie zum Beispiel Abschnitte, Bilder oder \LaTeX"=Blöcke.
  Elemente sind restrukturierbar und können weitere Elemente als Kinder enthalten. \\

  Expandable &
  Element mit ausklappbarem Inhalt, also ein Abschnitt oder Input. \\

  GUI, Frontend &
  Graphical User Interface, grafische Benutzeroberfläche.
  Im Browser sichtbarer Teil der Applikation. \\

  Inline-Makro &
  Ein Makro, das innerhalb eines Textabsatzes vorkommt.
  Es wird als Quelltext angezeigt, also semantisch ignoriert. \\

  Input &
  Ein Element basierend auf einer mit \verb|input| eingebundenen \LaTeX"=Datei. \\

  Kommentar &
  Vom User geschriebener Kommentar im Quelltext. \\

  Kompaktform &
  Eine Kurzzusammenfassung eines Elements, also zum Beispiel die Überschrift eines Abschnitts, der Name einer
  inkludierten Datei oder die ersten Worte eines Absatzes. \\

  \LaTeX"=Block &
  Ein Element, das einen nicht zu parsenden Teil des Dokuments darstellt, wie zum Beispiel Makros oder nicht
  unterstützte Umgebungen. \\

  Minieditor &
  Der Texteditor in der GUI, der es erlaubt, \LaTeX"=Quelltext von Strukturelementen zu bearbeiten. \\

  Notiz &
  Freitext-Zusatzinformation zu einem Strukturelement, meistens nicht sichtbar. \\

  REST-API &
  Representational State Transfer Application Programming Interface.
  Zustandslose Computerschnittstelle für die Kommunikation zwischen Frontend und Backend. \\

  Überschrift &
  Die Überschrift eines Abschnitts. \\

  Zusatzinformation, Attribut &
  An einem Strukturelement gespeicherte zusätzliche Informationen, die nicht zum \LaTeX"=Quelltext selbst gehören,
  zum Beispiel Notizen. \\
\end{longtable}
