\clearpage

\section{Motivation}
\label{sec:motivation}
Wissenschaftliche und technische Texte werden heute oftmals in \LaTeX{} geschrieben.
Dabei gehen die Autoren meist \enquote{top-down} vor,
\dh{} man überlegt sich zunächst die Grundstruktur und Überschriften und schreibt danach den eigentlichen Inhalt.
Konsequenz dieses Vorgehens ist, dass man das Dokument später im Schreibprozess noch einmal umstrukturieren will.
In herkömmlichen Anwendungen muss man hierfür per \textit{Copy and Paste} Abschnitte des \LaTeX"=Quelltextes bewegen.
Die Vielzahl an sichtbaren Makros und Syntaxelementen stört dabei den Lesefluss.
Daher ist es leicht, die Übersicht zu verlieren oder Teile eines zu verschiebenden Abschnittes zu vergessen.

Diese Probleme geht der grafische \LaTeX"=Editor \texla{} an.
Seine intuitive Benutzeroberfläche erlaubt es, Abschnitte, Absätze, Bilder und andere Blöcke einfach per
\textit{Drag and Drop} im Dokument hin und her zu verschieben und so das Dokument zu restrukturieren.
Dabei steht die Übersicht und die Orientierung im Dokument stets an erster Stelle.

Auch bei der Erstellung der Grundstruktur kann \texla{} die Arbeit der Schreibenden vereinfachen,
da auch Überschriften erzeugt werden können und live in einer Weise dargestellt werden,
die einem Inhaltsverzeichnis ähnelt.
Insgesamt soll \texla{} also den Aufbau und die Bearbeitung gut strukturierter Dokumente unterstützen.

